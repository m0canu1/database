\documentclass[12pt]{extarticle}
\usepackage{hyperref}
\usepackage[margin=1in]{geometry}
\usepackage{tabularx}
\begin{document}
\begin{titlepage}
\centering
{\LARGE\bfseries Basi di Dati}

\vspace{1cm}

{\Large Progetto d'esame}

\vspace{0.2cm}
{\large (a.a. 2018-2019)}

\vspace{2cm}

{\large Alexandru MOCANU}

\vspace{0.2cm}

{\small Matr. 813322}


\vspace{2cm}

% {\bfseries Submitted in fulfillment of the degree \ldots}

\vfill

{\itshape Università degli Studi di Torino - Dipartimento di Informatica}
\end{titlepage}

% Other content goes here...
\tableofcontents

\clearpage

\section{Progettazione concettuale}
\subsection{Requisiti Iniziali: SITO RICETTE}
Si vuole realizzare una base di dati per la gestione di una piattaforma in cui gli utenti possano
scoprire e condividere ricette\footnote{La piattaforma descritta è liberamente ispirata al sito Giallo Zafferano 
(\url{www.giallozafferano.it}), a cui è possibile fare riferimento per risolvere eventuali ambiguità nei requisiti}.
Per poter caricare le loro ricette, gli utenti devono registrarsi
fornendo \textbf{email}, \textbf{password} e un \textbf{nome utente} che sarà associato a tutti i loro contributi.
\\\\
Le ricette hanno un titolo, una descrizione, una foto di copertina, alcune informazioni su difficoltà,
tempo di preparazione, ecc. (si veda la Figura 1) ed una lista di ingredienti. Per ogni ingrediente si
specificano il nome, la quantità ed eventuali annotazioni. Inoltre, ogni ricetta appartiene ad una
categoria che identifica il tipo di portata (es. antipasto, primo piatto, …) e, in qualche caso, anche
ad una o più categorie che identificano la dieta associata (es. vegetariana, senza glutine, ...). Oltre
che dagli utenti, le ricette possono essere pubblicate dalla redazione del sito. Le ricette della
redazione possono avere come autore un cuoco famoso.
\\\\
Esplorando la pagina di un cuoco sul sito (si veda la Figura 2), i visitatori possono leggerne una
breve bio e accedere a tutte le ricette di cui questi è l’autore (di cui è subito visibile anche il
numero). Gli utenti possono seguire un cuoco per visualizzare le sue ricette in primo piano nella
propria pagina personale.
\\\\
Le persone possono inoltre salvare le ricette tra i propri preferiti e singoli ingredienti in una “lista
della spesa”. I preferiti e le cose da comprare sono visualizzati nella pagina personale degli utenti,
insieme ad un’indicazione del numero di ricette e del numero di commenti pubblicati.
\\\\
I visitatori possono infatti anche commentare le ricette, eventualmente rispondendo ad un
commento già pubblicato da qualcun altro. Anche la redazione può pubblicare dei commenti, ma
solo in risposta ai commenti lasciati dai visitatori. Per ogni commento vengono visualizzati: lo
username e la foto profilo dell’autore (o la dicitura “Redazione”), la data e il testo (si veda la Figura
3).
\\\\
Ad ogni ricetta sono associate delle ricette simili. Due ricette si considerano simili se hanno
almeno una categoria in comune e hanno come autore lo stesso cuoco o utente.
\\\\
Quando una persona cancella la propria iscrizione al sito, anche i suoi dati personali sono
cancellati, mentre le ricette che ha condiviso e i suoi commenti vengono mantenuti. Sono invece
cancellati insieme ai dati personali i preferiti, gli ingredienti contenuti nella lista della spesa, la lista
dei cuochi eventualmente seguiti.

\subsection{Glossario dei termini}
\vspace{2cm}
    \begin{tabularx}{17cm}{c|X|c|X}
    \bfseries Termine      & \bfseries Descrizione  & \bfseries Sinonimo  & \bfseries Collegamenti\\ 
    \hline\hline
    \bfseries Utente       & Persona che si iscrive al sito mediante un nome utente, email e password                                                                                                            & visitatore, persona & ricetta, ingrediuente, commento \\ 
    \hline
    \bfseries Ricetta      & Indicazione relativa agli ingredienti, alle dosi, alla preparazione di una pietanza, di una bevand o di un dolce.                                                              & /                   & utente, ingrediente, commento   \\
    \hline
    \bfseries Ingrediente  & Sostanze che entra a far parte di un prodotto  risulta necessaria all'esecuzione di una ricetta. Di ogni ingrediente viene rappresentato  il nome e la quantità da utilizzare & /                   & utente, ricetta                 \\
    \hline
    \bfseries Commento     & I commenti possono essere scritti da utenti sott una ricetta, oppure possono essere scritti in risposta ad un altro commento.                                                     & /                   & ricetta, utente                 \\
    \hline
    \bfseries Cuoco famoso & Un cuoco famoso ha un proprio profilo che può esser seguito dagli altri utenti.                                                                                                      & /                   & utente                          \\
    \hline
    \bfseries Redazione    & La Redazione si occupa della pubblicazione delle ricette  risponde ai commenti                                                                                                        & /                   & (ricetta, commento)             \\
    \end{tabularx}


\subsection{Requisiti rivisti e strutturati in gruppi di frasi omogenee}

\subsubsection{Requisiti riscritti}
\subsubsection{Strutturazione in gruppi di frasi omogenee}

\subsection{Schema E-R e Business Rules}
\subsubsection{Schema E-R}


\section{Progettazione logica}
\subsection{Tavola dei volumi}

\subsection{Tavola delle operazioni}

\subsection{Ristrutturazione Schema E-R}
\subsubsection{Analisi delle ridondanze}
\subsubsection{Eliminazione delle generalizzazioni}
\subsubsection{Scelta degli identificatori principali}

\subsection{Schema E-R ristrutturato e Business Rules}

\subsection{Schema Relazionale}

\section{Implementazione}
\subsection{DDL di creazione del Database}

\subsection{DML di popolamento del Database}

\subsection{Operazioni di Cancellazione e Modifica per la verifica dei vincoli e degli effetti causati da operazioni su chiavi esterne}


\end{document}