
\documentclass{exam}

\begin{document}
\printanswers

\section{Domande Teoria}

\begin{questions}
    \question Esprimere la condizione necessaria e sufficiente affinché una scomposizione sia senza perdita di informazione.
    \begin{solution}
    Sia R(A) uno schema con dipendenze funzionali F, decomposto in $\{R1(A1), R2(A2)\}$ dove $A_1 \cup A_2 = A$ \\
    La decomposizione di ogni istanza corretta(*) r(A) di  R(A) è senza perdita di informazione se e solo se  (condizione necessaria e sufficiente) \\  
    $(A1\cap A2$ superchiave di $A1)\vee(A1 \cap A2 $ superchiave di $A2) $
    \end{solution}
    \question Dire cos’è e quali problemi risolve il protocollo 2PL (due fasi) stretto.
    \question Dire in cosa consiste la tecnica del dump/restore (ripresa a freddo) e quando si rende necessaria.
    \question Esporre la differenza tra indici densi ed indici sparsi
    \question Indicare almeno \textbf{due casi} in cui gli indici secondari si dimostrano inefficienti
    \question Riportare la definizioe di chiusura di un insieme di attributi
    \question Spiegare la differenza tra ottimizzazione logica ed ottimizzazione fisica di un'interrogazione
    \question Riportare la definizione di protocollo 2PL (two-phase lock) stretto
    \question Dare la definizione di insieme di copertura minimale
    \question A proposito di gestione della concorrenza descrivere il protocollo di lock a due fasi (2PL) e quello di lock a due fasi stretto.
    \question Riportare la definizione di BCNF
    \question Elencare e spiegare brevemente le proprietà ACID delle transazioni
    \question Definire la nozione di azioni in conflitto in una storia S
    \question Descrivere il problema del deadlock avvalendosi del grafo di attesa
    \question Presentare una tecnica di \textbf{superamento} del deadlock
    \question Presentare una tecnica di \textbf{prevenzione} del deadlock
    \question Definire il concetto di dipendenza funzionale
    \question Enunciare il criterio di serializzabilità
    \question Differenze tra $B-tree$ e $B^{+}-tree$ e vantaggi dei $B^{+}-tree$ rispetto ai $B-tree$ nella gestione degli indici
    \question Mostrare un esempio semplice (una relazione R ed un insieme di dipendenze funzionali F) per cui  l’algoritmo di normalizzazione in BCNF non può essere in grado di mantenere la località delle dipendenze. 
    \question Che cosa si intende per \textbf{tupla spuria}
    \question Chiarire il concetto di decomposizione con join con o senza perdita di informazione
    \question Enunciare la condizione necessaria e sufficiente di decomponibilità senza perdita di informazione di una relazione in due sottorelazioni
    \question Enunciare il criteriodi view-serializzabilità di una storia
    \question Definire il grafo dei conflitti ed enunciare la condizione sufficiente di serializzabilità
    \question Indicare almeno tre casi in cui è preferibile evitare la definizione di indici secondari
\end{questions}



\paragraph{Dubbi}
\begin{enumerate}
    \item perdita della località delle dipendenze funzionali?
\end{enumerate}

\end{document}